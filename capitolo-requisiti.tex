\section{Requisiti }
I requisiti funzionali, prestazionali, di qualità e di vincolo individuati sono riportati nelle seguenti tabelle. Ogni requisito è identificato da un codice univoco.
Viene inoltre indicato se si tratta di un requisito fondamentale, desiderabile o facoltativo, una sua descrizione e il caso d'uso da cui è stato individuato. 

Ogni requisito è identificato da un codice, che segue il seguente formalismo:
\begin{center}
		\code{R\{X\}\{Y\}\{Z\} \{Gerarchia\}}
\end{center}

Dove:
\begin{itemize}
 \item \textbf{X} corrisponde al sistema di riferimento e può assumere i seguenti valori:
		\begin{itemize}
		 \item[] \textbf{A} = Applicazione \glossario{Maap};
		 \item[] \textbf{F} = \glossario{Framework MaaP};
		 \item[] \textbf{S} = \glossario{MaaS}.
		\end{itemize}

 \item \textbf{Y} corrisponde alla tipologia del requisito e può assumere i seguenti valori:
		\begin{itemize}
		 \item[] \textbf{1} = Funzionale;
		 \item[] \textbf{2} = Prestazionale;
		 \item[] \textbf{3} = Di Qualità;
		 \item[] \textbf{4} = Vincolo.
		\end{itemize}

 \item \textbf{Z} corrisponde alla priorità del requisito e può assumere i seguenti valori:
		\begin{itemize}
		 \item[] \textbf{O} = Obbligatorio
		 \item[] \textbf{D} = Desiderabile
		 \item[] \textbf{F} = Facoltativo o Opzionale
		\end{itemize}

 \item \textbf{Gerarchia} identifica la relazione gerarchica che c'è tra i requisiti di uno stesso tipo. C'è quindi una struttura gerarchica per ogni tipologia di requisito.
\end{itemize}

\subsection{Requisiti funzionali }

		%Tabella 
			\begin{center}
			\bgroup
			\def\arraystretch{1.8}
			\begin{longtable}{ | l | p{2cm} | p{5cm} | p{1.7cm} |}
		
			\cellcolor[gray]{0.9} \textbf{Requisito} & \cellcolor[gray]{0.9} \textbf{Tipologia} 
			& \cellcolor[gray]{0.9} \textbf{Descrizione} & \cellcolor[gray]{0.9} \textbf{Fonti} \\ \hline
      
				RA1O 1 & Funzionale \newline  Obbligatorio  & Il sistema permette all'utente non autenticato di autenticarsi tramite la visualizzazione di una pagina web, la quale conterrà al suo interno i campi di testo necessari.  &  UCS 0 \newline  UCU 4 \newline  UCU 4.1 \newline  \\ \hline      
				RA1O 1.1 & Funzionale \newline  Obbligatorio  & Il sistema prevede l'inserimento dell'indirizzo email per la verifica delle credenziali in un apposito campo di testo. &  \\ \hline      
				RA1O 1.2 & Funzionale \newline  Obbligatorio  & Il sistema prevede l'inserimento di password per la verifica delle credenziali in un apposito campo di testo. &  UCU 4.1 \newline  \\ \hline      
				RA1O 1.3 & Funzionale \newline  Obbligatorio  & Il sistema, tramite un database indipendente, ovvero separato da quello che contiene la Collection, provvede a verificare l'autenticità  di un utente tramite la verifica di email e password. &  \\ \hline      
				RA1O 1.3.1 & Funzionale \newline  Obbligatorio  & Il sistema mette a disposizione la visualizzazione di una pagina di errore in caso di fallimento dell'autenticazione da parte dell'utente. &  \\ \hline      
				RA1O 1.3.2 & Funzionale \newline  Obbligatorio  & Il sistema, nel caso in cui l'autenticazione da parte dell'utente abbia avuto successo, reindirizza automaticamente l'utente sulla dashboard dell'applicazione. &  \\ \hline      
				RA1O 2.1.5.9 & Funzionale \newline  Obbligatorio  & prova &  \\ \hline      
				RA1O 4.2 & Funzionale \newline  Obbligatorio  & prova &  \\ \hline      
				RA1O 5.1 & Funzionale \newline  Obbligatorio  & prova &  \\ \hline      
				RA1O 5.10 & Funzionale \newline  Obbligatorio  & prova &  \\ \hline
			\caption{Requisiti funzionali}
			\end{longtable}
			\egroup
			\end{center}  
\clearpage

\subsection{Requisiti di qualità }

		%Tabella 
			\begin{center}
			\bgroup
			\def\arraystretch{1.8}
			\begin{longtable}{ | l | p{2cm} | p{5cm} | p{1.7cm} |}
		
			\cellcolor[gray]{0.9} \textbf{Requisito} & \cellcolor[gray]{0.9} \textbf{Tipologia} 
			& \cellcolor[gray]{0.9} \textbf{Descrizione} & \cellcolor[gray]{0.9} \textbf{Fonti} \\ \hline
      
				R3O 1 & Qualità \newline  Obbligatorio  & Devono essere prodotti e rilasciati manuali d'uso ed ogni altra documentazione tecnica necessaria per l’utilizzo del prodotto. &  \\ \hline
			\caption{Requisiti di qualità}
			\end{longtable}
			\egroup
			\end{center}  
\clearpage

\subsection{Requisiti di vincolo }

		%Tabella 
			\begin{center}
			\bgroup
			\def\arraystretch{1.8}
			\begin{longtable}{ | l | p{2cm} | p{5cm} | p{1.7cm} |}
		
			\cellcolor[gray]{0.9} \textbf{Requisito} & \cellcolor[gray]{0.9} \textbf{Tipologia} 
			& \cellcolor[gray]{0.9} \textbf{Descrizione} & \cellcolor[gray]{0.9} \textbf{Fonti} \\ \hline
      
				RA4O 1 & Vincolo \newline  Obbligatorio  & L’implementazione della componente server deve essere realizzata utilizzando Node.js. &  \\ \hline      
				R4O 1 & Vincolo \newline  Obbligatorio  & Deve essere utilizzato Express per la realizzazione dell’infrastruttura della web application. &  Capitolato \newline  Interno \newline  Verbale-2013-12-05 \newline  Verbale-2013-12-18 \newline  UCU 0 \newline  \\ \hline
			\caption{Requisiti di vincolo}
			\end{longtable}
			\egroup
			\end{center}  
\clearpage
\section{Tracciamento Requisiti}
\subsection{Tracciamento requisiti-fonti}
%Tabella 
			\begin{center}
			\bgroup
			\def\arraystretch{1.8}
			\begin{longtable}{ | p{5cm} | p{5cm} |}
		
			\cellcolor[gray]{0.9} \textbf{Requisiti} & \cellcolor[gray]{0.9} \textbf{Fonti} \\ \hline       
				RA1O 1 &  UCS 0 \newline  UCU 4 \newline  UCU 4.1 \newline  \\ \hline      
				RA1O 1.1 &  \\ \hline      
				RA1O 1.2 &  UCU 4.1 \newline  \\ \hline      
				RA1O 1.3 &  \\ \hline      
				RA1O 1.3.1 &  \\ \hline      
				RA1O 1.3.2 &  \\ \hline      
				RA1O 2.1.5.9 &  \\ \hline      
				RA1O 4.2 &  \\ \hline      
				RA1O 5.1 &  \\ \hline      
				RA1O 5.10 &  \\ \hline      
				R3O 1 &  \\ \hline      
				RA4O 1 &  \\ \hline      
				R4O 1 &  Capitolato \newline  Interno \newline  Verbale-2013-12-05 \newline  Verbale-2013-12-18 \newline  UCU 0 \newline  \\ \hline  
			\caption{Tracciamento requisiti-fonti}    
			\end{longtable}
			\egroup
			\end{center}  
\clearpage

\subsection{Tracciamento fonti-requisiti}
%Tabella 
			\begin{center}
			\bgroup
			\def\arraystretch{1.8}
			\begin{longtable}{ | p{5cm} | p{5cm} |}
		
			\cellcolor[gray]{0.9} \textbf{Fonti} & \cellcolor[gray]{0.9} \textbf{Requisiti} \\ \hline       
						UCU 1 - Login &  \\ \hline      
						UCU 2 - Visualizzazione messaggio errore per credenziali errate &  \\ \hline      
						UCU 4 - Recupero password &  RA1O 1 \newline  \\ \hline      
						UCU 4.1 - Richiesta reset password &  RA1O 1 \newline  RA1O 1.2 \newline  \\ \hline      
						UCS 0 - Operazioni ad alto livello - Utente MaaS non autenticato &  RA1O 1 \newline  \\ \hline  
			\caption{Tracciamento fonti-requisiti}   
			\end{longtable}
			\egroup
			\end{center}  
\clearpage
\subsection{Tracciamento Requisiti - Test di Sistema e Validazione}

	\begin{center}
	\def\arraystretch{1.5}
	\bgroup
		\begin{longtable}{| p{2cm} | p{6cm} | p{2.5cm} | p{2.5cm} | }
		\hline 
		 \textbf{Requisito} & \textbf{Descrizione} & \textbf{Test di Sistema} & \textbf{Test di Validazione} \\ \hline
				RA1O 1 & 
				Il sistema permette all'utente non autenticato di autenticarsi tramite la visualizzazione di una pagina web, la quale conterrà al suo interno i campi di testo necessari.  & TS-RA1O 1.1 & \\ \hline 
				RA1O 1.1 & 
				Il sistema prevede l'inserimento dell'indirizzo email per la verifica delle credenziali in un apposito campo di testo. & TS-RA1O 1.1 & \\ \hline 
				RA1O 1.2 & 
				Il sistema prevede l'inserimento di password per la verifica delle credenziali in un apposito campo di testo. & TS-RA1O 1.1 &TV-RA1O 1.2 \\ \hline 
				RA1O 1.3 & 
				Il sistema, tramite un database indipendente, ovvero separato da quello che contiene la Collection, provvede a verificare l'autenticità  di un utente tramite la verifica di email e password. &  &  \\ \hline 
				RA1O 1.3.1 & 
				Il sistema mette a disposizione la visualizzazione di una pagina di errore in caso di fallimento dell'autenticazione da parte dell'utente. &  &  \\ \hline 
				RA1O 1.3.2 & 
				Il sistema, nel caso in cui l'autenticazione da parte dell'utente abbia avuto successo, reindirizza automaticamente l'utente sulla dashboard dell'applicazione. &  &  \\ \hline 
				RA1O 2.1.5.9 & 
				prova & TS-RA1O 2.1.5.9 &TV-RA1O 2.1.5.9 \\ \hline 
				RA1O 4.2 & 
				prova &  &  \\ \hline 
				RA1O 5.1 & 
				prova &  &  \\ \hline 
				RA1O 5.10 & 
				prova &  & TV-RA1O 5.10 \\ \hline 
		\caption{Tracciamento Requisiti - Test di Sistema e Validazione}
		\end{longtable}
	 \egroup
\end{center}
\clearpage

\subsection{Tracciamento Requisiti Accettati - Test di Sistema e Validazione}

	\begin{center}
	\def\arraystretch{1.5}
	\bgroup
		\begin{longtable}{| p{2cm} | p{6cm} | p{2.5cm} | p{2.5cm} | }
		\hline 
		 \textbf{Requisito} & \textbf{Descrizione} & \textbf{Test di Sistema} & \textbf{Test di Validazione} \\ \hline
					RA1O 1 & 
					Il sistema permette all'utente non autenticato di autenticarsi tramite la visualizzazione di una pagina web, la quale conterrà al suo interno i campi di testo necessari.  & TS-RA1O 1.1 & \\ \hline 
					RA1O 1.1 & 
					Il sistema prevede l'inserimento dell'indirizzo email per la verifica delle credenziali in un apposito campo di testo. & TS-RA1O 1.1 & \\ \hline 
					RA1O 1.2 & 
					Il sistema prevede l'inserimento di password per la verifica delle credenziali in un apposito campo di testo. & TS-RA1O 1.1 &TV-RA1O 1.2 \\ \hline 
					RA1O 1.3 & 
					Il sistema, tramite un database indipendente, ovvero separato da quello che contiene la Collection, provvede a verificare l'autenticità  di un utente tramite la verifica di email e password. &  &  \\ \hline 
					RA1O 1.3.1 & 
					Il sistema mette a disposizione la visualizzazione di una pagina di errore in caso di fallimento dell'autenticazione da parte dell'utente. &  &  \\ \hline 
					RA1O 1.3.2 & 
					Il sistema, nel caso in cui l'autenticazione da parte dell'utente abbia avuto successo, reindirizza automaticamente l'utente sulla dashboard dell'applicazione. &  &  \\ \hline 
		\caption{Tracciamento Requisiti - Test di Sistema e Validazione}
		\end{longtable}
	 \egroup
\end{center}
\clearpage
